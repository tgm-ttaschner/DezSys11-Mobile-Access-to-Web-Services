%!TEX root=../document.tex

\section{Einführung}
Diese Übung gibt einen Einblick in Entwicklungen von mobilen Applikationen.

\subsection{Ziele}
Das Ziel dieser Übung ist eine Anbindung einer mobilen Applikation an ein Webservices.

Die Anbindung soll mit Hilfe eines RESTful Webservice (Gruppe1) umgesetzt werden.


\subsection{Voraussetzungen}
\begin{itemize}
\item Grundlagen Java und XML
\item Grundlegendes Verständnis über Entwicklungs- und Simulationsumgebungen
\item Verständnis von RESTful Webservices
\end{itemize}


\subsection{Aufgabenstellung}
Es ist eine mobile Anwendung zu implementieren, die sich an das Webservice aus der Übung DezSysLabor-09 "Web Services in Java" anbinden soll. Dabei müssen die entwickelten Schnittstellen entsprechend angesprochen werden.

Es ist freigestellt, welche mobile Implementierungsumgebung dafür gewählt wird. Empfohlen wird aber eine Implementierung auf Android.
\newline

\uline{Bewertung:} 16 Punkte
\begin{list}{-}
\item Anbindung einer mobilen Applikation an die Webservice-Schnittstelle (6 Punkte)
\item Registrierung von Benutzern (3 Punkte)
\item Login und Anzeige einer Willkommensnachricht (3 Punkte)
\item Simulation bzw. Deployment auf mobilem Gerät (2 Punkte)
\item Protokoll (2 Punkte)
\end{list}

\section{Quellen}
\label{sec:Quellen}
\cite{RESTWebservice, DezSys09Paul}

\clearpage
